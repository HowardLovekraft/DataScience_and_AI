\input{preamble-article}

\begin{document}
	\title{Работа для 2 практики}
	\author{Холкин Николай}
	\date{16 сентября 2025 г.}
	
	\maketitle
	
		% Шаблон не соответствует ГОСТу 7.32
		% За шаблоном по ГОСТу 7.32 - сюда:
		% https://github.com/latex-g7-32/latex-g7-32/releases/tag/5.0.0
		
		
		\subsection{Предметная область}
		Матчи в Dota 2
		
	\subsection{Гипотеза}
		Совершение первого убийства в игре влияет на вероятность победы команды.
		
	\subsection{Необходимый объем данных}
		2 датасета, $~10000+$ сэмплов
		
	\subsection{Источники сбора данных}
		\begin{itemize}
			\item \href{https://www.kaggle.com/datasets/bwandowando/dota-2-pro-league-matches-2023}{Kaggle}
			\item \href{https://archive.ics.uci.edu/dataset/367/dota2+games+results}{UCI MLR}
		\end{itemize}
		
	Были выбраны такие варианты сбора данных, т.к. задача не новая, в ней уже есть качественные датасеты, собранные сообществом.
	
	\paragraph{Объем собранных данных}
		\begin{enumerate}
			\item Датасет с Kaggle:
			\begin{itemize}
				\item 12 csv-таблиц, отражающих каждый аспект матча
			\end{itemize}
			
			\item Датасет с UCI:
			\begin{itemize}
				\item 115 фичей
				\item 102000 сэмплов
			\end{itemize}
		\end{enumerate}
		
		Природа данных естественная, датасет был собран путем парсинга сайта со статистикой Dota 2.
	
	\subsection{План для анализа данных}
		\begin{enumerate}
			\item Проверить гипотезу
			\item Выявить признаки, указывающую на победу
			\item Написать аглоритм, который будет предсказывать исход матча по параметрам игроков.
		\end{enumerate}
\end{document}
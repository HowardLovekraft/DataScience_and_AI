\documentclass[12pt]{article}

\include{cmap}
\usepackage[unicode]{hyperref}

\usepackage{amsmath, amsthm, amssymb, amsfonts}
\usepackage{mathtext, mathtools}
\usepackage{xcolor, listings}
\usepackage{algorithm2e}
\usepackage{pgfplots}

\usepackage[T1, T2A]{fontenc}
\usepackage[utf8]{inputenc}
\usepackage[english, russian]{babel}

\usepackage{indentfirst}
\usepackage{paracol}

\usepackage{graphicx}
\usepackage{geometry}
\geometry{a4paper,
	total={170mm,257mm},left=2cm,right=2cm,
	top=2cm,bottom=2cm}

\usepackage{titleps}
\newpagestyle{main}{
	\setheadrule{0.4pt}
	\sethead{лево}{центр}{право}
	\setfootrule{0.4pt}
	\setfoot{left}{\thepage}{право}
}

\DeclareMathOperator{\sign}{sign}
\DeclareMathOperator{\sigmoid}{sigmoid}
\DeclareMathOperator*{\argmax}{argmax}

\theoremstyle{plain}
\newtheorem{theorem}{Теорема}[section]
\newtheorem{corollary}{Следствие}[theorem]
\newtheorem*{definition}{Определение}

\newenvironment{eq_array}{\begin{equation*}\begin{array}{l}}{\end{array}\end{equation*}}

\lstset{
	extendedchars=false,
	language=Python,                 % выбор языка для подсветки (здесь это Python)
	basicstyle=\small\sffamily, % размер и начертание шрифта для подсветки кода
	numbers=left,               % где поставить нумерацию строк (слева\справа)
	numberstyle=\tiny,           % размер шрифта для номеров строк
	stepnumber=1,                   % размер шага между двумя номерами строк
	numbersep=5pt,                % как далеко отстоят номера строк от подсвечиваемого кода
	backgroundcolor=\color{white}, % цвет фона подсветки - используем \usepackage{color}
	showspaces=false,            % показывать или нет пробелы специальными отступами
	showstringspaces=false,      % показывать или нет пробелы в строках
	showtabs=false,             % показывать или нет табуляцию в строках
	frame=single,              % рисовать рамку вокруг кода
	tabsize=2,                 % размер табуляции по умолчанию равен 2 пробелам
	captionpos=t,              % позиция заголовка вверху [t] или внизу [b] 
	breaklines=true,           % автоматически переносить строки (да\нет)
	breakatwhitespace=false, % переносить строки только если есть пробел
	escapeinside={\%*}{*)},   % если нужно добавить комментарии в коде,
	keepspaces=true
}


\begin{document}
	\title{Работа для 2 практики}
	\author{Холкин Николай}
	\date{16 сентября 2025 г.}
	
	\maketitle
	
		% Шаблон не соответствует ГОСТу 7.32
		% За шаблоном по ГОСТу 7.32 - сюда:
		% https://github.com/latex-g7-32/latex-g7-32/releases/tag/5.0.0
		
		
	\section{Предметная область}
		Вселенная <<Песни льда и пламени>> (Игры престолов)
		
	\subsection{Гипотеза}
		Персонажи, появившиеся в последних главах, имеют больше шансов
		на выживание, чем персонажи из ранних глав.
	\subsection{Необходимый объем данных}
		$\approx$ 1000-1500 сэмплов
		
	\subsection{Источники сбора данных}
		\begin{itemize}
			\item \href{https://www.kaggle.com/competitions/got-predicting-characters-deaths/}{Kaggle}
		\end{itemize}
		
	Можно выбрать только один источник данных - выгрузку парсинга с вики. В основе \href{http://awoiaf.westeros.org}{вики} лежит информация из книг Джорджа Мартина, каким-либо другим способом получить информацию о персонажах не получится. 
	
	\subsection {Объем собранных данных}
		Датасет с Kaggle:
		\begin{itemize}
			\item 21 параметр
			\item 1946 объектов
		\end{itemize}
		
		Природа данных естественная, датасет был собран путем парсинга вики фандома, википедия заполнялась по информации из первоисточников.
	
	\subsection{План для анализа данных}
		\begin{enumerate}
			\item Проверить гипотезу
			\item Выявить признаки, указывающую на смерть персонажа:
			\begin{enumerate}
				\item Раннее появление в серии повышает вероятность смерти?
				\item <<Высокая>> связанность с другими мёртвыми персонажами повышает вероятность смерть?
			\end{enumerate}
			\item Выявить признаки, указывающие на выживание персонажа
			\begin{enumerate}
				\item Высокое соц. положение повышает шансы выживания?
			\end{enumerate}
			\item Обучить алгоритм, который будет предсказывать судьбу персонажа по его характеристикам
		\end{enumerate}
\end{document}
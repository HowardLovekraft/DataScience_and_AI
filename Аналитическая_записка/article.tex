\input{preamble-article}

\begin{document}
	\title{Работа для 2 практики}
	\author{Холкин Николай}
	\date{16 сентября 2025 г.}
	
	\maketitle
	
		% Шаблон не соответствует ГОСТу 7.32
		% За шаблоном по ГОСТу 7.32 - сюда:
		% https://github.com/latex-g7-32/latex-g7-32/releases/tag/5.0.0
		
		
	\section{Предметная область}
		Вселенная <<Песни льда и пламени>> (Игры престолов)
		
	\subsection{Гипотеза}
		Персонажи, появившиеся в последних главах, имеют больше шансов
		на выживание, чем персонажи из ранних глав.
	\subsection{Необходимый объем данных}
		$\approx$ 1000-1500 сэмплов
		
	\subsection{Источники сбора данных}
		\begin{itemize}
			\item \href{https://www.kaggle.com/competitions/got-predicting-characters-deaths/}{Kaggle}
		\end{itemize}
		
	Можно выбрать только один источник данных - выгрузку парсинга с вики. В основе \href{http://awoiaf.westeros.org}{вики} лежит информация из книг Джорджа Мартина, каким-либо другим способом получить информацию о персонажах не получится. 
	
	\subsection {Объем собранных данных}
		Датасет с Kaggle:
		\begin{itemize}
			\item 21 параметр
			\item 1946 объектов
		\end{itemize}
		
		Природа данных естественная, датасет был собран путем парсинга вики фандома, википедия заполнялась по информации из первоисточников.
	
	\subsection{План для анализа данных}
		\begin{enumerate}
			\item Проверить гипотезу
			\item Выявить признаки, указывающую на смерть персонажа:
			\begin{enumerate}
				\item Раннее появление в серии повышает вероятность смерти?
				\item <<Высокая>> связанность с другими мёртвыми персонажами повышает вероятность смерть?
			\end{enumerate}
			\item Выявить признаки, указывающие на выживание персонажа
			\begin{enumerate}
				\item Высокое соц. положение повышает шансы выживания?
			\end{enumerate}
			\item Обучить алгоритм, который будет предсказывать судьбу персонажа по его характеристикам
		\end{enumerate}
\end{document}